\begin{abstract}

SAFE is a tool that enables continuous assessment in the form of regular/weekly quizzes in classes. SAFE is based on a BYOD (bring your own device) model that leverages student smartphones to conduct auto-graded, cheating-free exams in a proctored class room setting. SAFE has 3 components: a smartphone app, a web server and WiFi infrastructure to enable app-server communication. The platform has so far been used in 130+ in class quizzes across 9 courses. It was also used to conduct a high stake admission test for a Master’s program in Computer Science.\\

The currently used PHP based server and Android app was initially developed as proof of concept and later expanded to include more features and bug fixes as more people started using the platform. Statistics for quizzes was added to know how students performed in the quiz overall as well as question wise statistics on how many students answered correctly, incorrectly and did not attempt. This feature allows the instructor to know the overall performance of the students and spend more time on the topics that had bad performance. Adding support for code sinppets in the question text was another requirement that was implemented. There was also a requirement for generating a print view of the questions with or without the answers so that instructors could conduct a written quiz parallely, or discuss the questions at a later point of time. \\

As time progressed it became evident that a complete rewrite of the code was required in order to make the code modular, scalable, maintainable and to improve performance. After cosiderable research it was decided to use python based Django framework for the backend, along with PostgreSQL database and Celery for asynchronous tasks. The Android app was also rewritten for better code maintainability and performance. During the time of starting of the project both the Django server and Android app rewrite had just been completed with the bare minimum features to start testing. The first major task was to properly test the newly written code and fix any bugs that was found out in the process. \\

Once the app was stable the next task was to create a docker container to deploy the server in both production and development modes. Prior to this the time required to setup the SAFE server on a fresh machine was around 4 hours. By creating a docker container with all the requirements inside it, the time required for setting up the server on a machine came down to 30 minutes (human intervention is required only in the initial 5 minutes, rest of 25 minutes are taken for downloading of the required software and their installation) \\
   
It was found that many instructors using the platform for the first time had difficulty in understanding the platforms and setting up course / quiz by themselves. There a new UI was prototyped with the latest UI / UX best practices and patterns, to make the system more intuitive and easier to use.
\end{abstract}